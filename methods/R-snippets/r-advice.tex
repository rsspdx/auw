% Options for packages loaded elsewhere
\PassOptionsToPackage{unicode}{hyperref}
\PassOptionsToPackage{hyphens}{url}
\PassOptionsToPackage{dvipsnames,svgnames,x11names}{xcolor}
%
\documentclass[
  letterpaper,
  DIV=11,
  numbers=noendperiod]{scrartcl}

\usepackage{amsmath,amssymb}
\usepackage{lmodern}
\usepackage{iftex}
\ifPDFTeX
  \usepackage[T1]{fontenc}
  \usepackage[utf8]{inputenc}
  \usepackage{textcomp} % provide euro and other symbols
\else % if luatex or xetex
  \usepackage{unicode-math}
  \defaultfontfeatures{Scale=MatchLowercase}
  \defaultfontfeatures[\rmfamily]{Ligatures=TeX,Scale=1}
\fi
% Use upquote if available, for straight quotes in verbatim environments
\IfFileExists{upquote.sty}{\usepackage{upquote}}{}
\IfFileExists{microtype.sty}{% use microtype if available
  \usepackage[]{microtype}
  \UseMicrotypeSet[protrusion]{basicmath} % disable protrusion for tt fonts
}{}
\makeatletter
\@ifundefined{KOMAClassName}{% if non-KOMA class
  \IfFileExists{parskip.sty}{%
    \usepackage{parskip}
  }{% else
    \setlength{\parindent}{0pt}
    \setlength{\parskip}{6pt plus 2pt minus 1pt}}
}{% if KOMA class
  \KOMAoptions{parskip=half}}
\makeatother
\usepackage{xcolor}
\setlength{\emergencystretch}{3em} % prevent overfull lines
\setcounter{secnumdepth}{-\maxdimen} % remove section numbering
% Make \paragraph and \subparagraph free-standing
\ifx\paragraph\undefined\else
  \let\oldparagraph\paragraph
  \renewcommand{\paragraph}[1]{\oldparagraph{#1}\mbox{}}
\fi
\ifx\subparagraph\undefined\else
  \let\oldsubparagraph\subparagraph
  \renewcommand{\subparagraph}[1]{\oldsubparagraph{#1}\mbox{}}
\fi


\providecommand{\tightlist}{%
  \setlength{\itemsep}{0pt}\setlength{\parskip}{0pt}}\usepackage{longtable,booktabs,array}
\usepackage{calc} % for calculating minipage widths
% Correct order of tables after \paragraph or \subparagraph
\usepackage{etoolbox}
\makeatletter
\patchcmd\longtable{\par}{\if@noskipsec\mbox{}\fi\par}{}{}
\makeatother
% Allow footnotes in longtable head/foot
\IfFileExists{footnotehyper.sty}{\usepackage{footnotehyper}}{\usepackage{footnote}}
\makesavenoteenv{longtable}
\usepackage{graphicx}
\makeatletter
\def\maxwidth{\ifdim\Gin@nat@width>\linewidth\linewidth\else\Gin@nat@width\fi}
\def\maxheight{\ifdim\Gin@nat@height>\textheight\textheight\else\Gin@nat@height\fi}
\makeatother
% Scale images if necessary, so that they will not overflow the page
% margins by default, and it is still possible to overwrite the defaults
% using explicit options in \includegraphics[width, height, ...]{}
\setkeys{Gin}{width=\maxwidth,height=\maxheight,keepaspectratio}
% Set default figure placement to htbp
\makeatletter
\def\fps@figure{htbp}
\makeatother

\KOMAoption{captions}{tableheading}
\makeatletter
\makeatother
\makeatletter
\makeatother
\makeatletter
\@ifpackageloaded{caption}{}{\usepackage{caption}}
\AtBeginDocument{%
\ifdefined\contentsname
  \renewcommand*\contentsname{Table of contents}
\else
  \newcommand\contentsname{Table of contents}
\fi
\ifdefined\listfigurename
  \renewcommand*\listfigurename{List of Figures}
\else
  \newcommand\listfigurename{List of Figures}
\fi
\ifdefined\listtablename
  \renewcommand*\listtablename{List of Tables}
\else
  \newcommand\listtablename{List of Tables}
\fi
\ifdefined\figurename
  \renewcommand*\figurename{Figure}
\else
  \newcommand\figurename{Figure}
\fi
\ifdefined\tablename
  \renewcommand*\tablename{Table}
\else
  \newcommand\tablename{Table}
\fi
}
\@ifpackageloaded{float}{}{\usepackage{float}}
\floatstyle{ruled}
\@ifundefined{c@chapter}{\newfloat{codelisting}{h}{lop}}{\newfloat{codelisting}{h}{lop}[chapter]}
\floatname{codelisting}{Listing}
\newcommand*\listoflistings{\listof{codelisting}{List of Listings}}
\makeatother
\makeatletter
\@ifpackageloaded{caption}{}{\usepackage{caption}}
\@ifpackageloaded{subcaption}{}{\usepackage{subcaption}}
\makeatother
\makeatletter
\@ifpackageloaded{tcolorbox}{}{\usepackage[many]{tcolorbox}}
\makeatother
\makeatletter
\@ifundefined{shadecolor}{\definecolor{shadecolor}{rgb}{.97, .97, .97}}
\makeatother
\makeatletter
\makeatother
\ifLuaTeX
  \usepackage{selnolig}  % disable illegal ligatures
\fi
\IfFileExists{bookmark.sty}{\usepackage{bookmark}}{\usepackage{hyperref}}
\IfFileExists{xurl.sty}{\usepackage{xurl}}{} % add URL line breaks if available
\urlstyle{same} % disable monospaced font for URLs
\hypersetup{
  pdftitle={Advice on using R},
  pdfauthor={Richard Sherman},
  colorlinks=true,
  linkcolor={blue},
  filecolor={Maroon},
  citecolor={Blue},
  urlcolor={Blue},
  pdfcreator={LaTeX via pandoc}}

\title{Advice on using R}
\author{Richard Sherman}
\date{}

\begin{document}
\maketitle
\ifdefined\Shaded\renewenvironment{Shaded}{\begin{tcolorbox}[sharp corners, boxrule=0pt, interior hidden, borderline west={3pt}{0pt}{shadecolor}, enhanced, frame hidden, breakable]}{\end{tcolorbox}}\fi

\hypertarget{first-steps}{%
\section{First steps}\label{first-steps}}

You should have \textsf{R} and \textsf{RStudio} installed on your
computer. It is installed in our labs. To install it on your computer:

\begin{itemize}
\tightlist
\item
  \textsf{R} https://cran.r-project.org (Mac/Linux/Windows)
\item
  \textsf{RStudio} https://posit.co/download/rstudio-desktop/
  (Mac/Linux/Windows)
\item
  For Chromebook:
  https://blog.sellorm.com/2018/12/20/installing-r-and-rstudio-on-a-chromebook/
  (installs both)
\end{itemize}

\hypertarget{do-things-sensibly}{%
\subsection{Do things sensibly}\label{do-things-sensibly}}

Create a folder (directory) on your computer and/or your computer lab
account to keep all your work and data for the course. Something simple
like `auw/methods' is a good idea. (Note: in computer-speak, the
shortcut `\textasciitilde/' refers to your home directory, so the path
to this directory would be `\textasciitilde/auw/methods'). On my
computer, the path I use is `\textasciitilde/data/auw'.

When you start \textsf{R}, you can find out where you are in your
computer with the command:

\texttt{getwd}

(\texttt{wd}) stands for `working directory' Probably, you are not where
you want to be. You should change the working directory to your chosen
place for your work in the course, perhaps
`\textasciitilde/auw/methods', or in my case `\textasciitilde/data/auw'.
Note that you need the quotation marks:

\texttt{setwd('~/data/auw')}

\textsf{R}, like most computer languages, relies on external libraries
(`packages', in \textsf{R}-speak). We will make extensive use of the
`\textsf{tidyverse}' package. You need to install this:

\texttt{install.packages('tidyverse')}

You will probably get a pop-up box asking you to choose a `mirror', a
place where \textsf{R} packages are located. Just choose one close to
your location. India is the closest, I think, but you can use whichever
one you like. The speed of light means that `closeness' doesn't matter
much.

You should only have to install the packages once. However, to
\emph{use} the package, you need to call it at the beginning of your
\textsf{R} file. So a standard way to get things started would be, after
installing \textsf{tidyverse}:

\begin{enumerate}
\def\labelenumi{\arabic{enumi}.}
\tightlist
\item
  Start RStudio
\item
  Go to File -\textgreater{} New File -\textgreater{} R script
\item
  Go to File -\textgreater{} Save and save your new R script in your
  auw/methods folder
\item
  At the top of your new R script, write a comment explaining what the
  script is for. Comments in \textsf{R} begin with the character
  \verbatim{#}. You might write something like
  \verbatim{# An Example \textsf{R} Script}.
\end{enumerate}

An Example \textsf{R} script:

\texttt{ this is an example of how you might begin a script}

\texttt{install.packages('tidyverse')} (if you haven't done this before)

Find out where you are on your computer using the R command:

\texttt{getwd()}

Get to where you want to by using this command:

\texttt{setwd('~/data/auw')} (note the quotation marks)

After this, you can add other commands to begin doing some analysis.
Follow the examples in the Wickham text
(\emph{\textsf{R} for Data Science)}. Then use one of the snippets in
our course folder. You might want to copy the course folder to your own
computer.

Enjoy!



\end{document}
